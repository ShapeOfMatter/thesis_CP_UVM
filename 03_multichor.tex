\chapter{Real World Choreographic Programming}
\chaptermark{\MultiChor?} %this is the chapter heading that will show on subsequent pages
\label{sec:multichor}

\begin{quote}
Paper abstract
\end{quote}


\section{Introduction}
We should expect this to be a mashup of the original \MultiChor paper and the enclaves paper.
That said, I'd kinda like to include "MiniChor" (\MultiChor with fancy MLVs, simplified \inlinecode{Locally}, and no \inlinecode{Congruently}),
either as an aside or as the textbook implementation.

\subsection{Implementation Details}

A random citation as an example \shortcite{krook2024parti}.

During the job interview with Well Typed, they discussed some alternatives to the proof-witness pattern that can probably work.
I haven't played with it enough to say if it would be \emph{better},
but I should make sure to understand what they're doing well enough to express an opinion about it here.

\section{Census Polymorphism}
\label{sec:census-poly}

TODO: This is copy-pasted directly from the enclaves paper; it must be edited/rewritten for this context.

So far, the example choreographies we have discussed have had fixed numbers of participants.
In all prior CP systems this has been a syntactic constraint:
even systems that allow polymorphism over the identities of participants require the participants' "roles" to be explicitly defined in-context.
This is a serious limitation for writing choreographic software;
modern concurrent systems often use dozens to thousands of participants
and are defined parametrically over their number of participants~\cite{bigConcurrent1, corrigan2017prio, bigConcurrent3, bigConcurrent4, dprio2023}.
We assert that such parametric protocol declarations are a required feature for CP to find mainstream use;
our systems provide it in the form of \emph{census polymorphism}.

By "census polymorphism", we mean that a choreographic expression is polymorphic over its census type-variable,
including not just the specific identities listed but also the quantity.\footnote{
    In principle, one can split hairs between census polymorphism and similar polymorphism over other sets of parties, \eg ownership sets.
    We have not found such distinctions to be useful for describing system capabilities,
    but they can be relevant when talking about the type of a given expression.
}
Naïvely, this is trivial; any \MultiChor expression can easily be written with a type variable as its census
and the relevant parties (whose exact identities can also be polymorphic) can be guaranteed to be present by taking
membership proofs as arguments.
However, this approach has a limitation: Since the number of type variables of a choreography must be fixed
and there is no way to \emph{explicitly list} an arbitrary number of parties,
there may be parties not identified by such arguments.
They'll receive any broadcasts and participate in any active replication that applies to the whole census,
but there's no way to specify them as the senders of messages, nor is there any way to specify that they should receive a message
except by sending it to the whole census.
For this reason, when we speak of "census polymorphism",
we mean \emph{useful} polymorphism that lets an unspecified quantity of parties actively participate in the choreography.
For example one might wish to write a \inlinecode{gather} operation
in which a polymorphic list of participants each send a computed value to a common recipient who aggregates them.
Figure~\ref{fig:census-poly-example} shows an example \MultiChor choreography for a key-value store with a polymorphic list of backup servers.
In Appendix~\ref{sec:mpc} we implement the GMW protocol~\cite{goldreich2019play}, a foundational protocol in multi-party cryptography.
In earlier CP systems it would be necessary to hard-code the number of participants when writing such choreographies.
In this section we show how our libraries overcome the above limitation and represent census-polymorphic choreographies.

\begin{figure*}[tbhp]
  \begin{mdframed}
    \inputminted[xleftmargin=10pt,linenos,fontsize=\scriptsize]{haskell}{figures/kvs_poly.hs.txt}
    \caption{
        A key-value store choreography with an unspecified number of backup servers.
        The main action happens in \inlinecode{handleRequest},
        a choreography involving only the servers which is called via \inlinecode{enclave} on line 21.
        \inlinecode{handleRequest}'s census explicitly includes the primary server, but is polymorphic over the list of backup servers.
        The primary server broadcasts the request (line~5--12); the backups will update their state and report their health
        only for a \inlinecode{Put} request.
        On line 7 the backups call the local IO function \inlinecode{handlePut} in \inlinecode{parallel} using their individual state references;
        \inlinecode{oks} is therefore a \inlinecode{Faceted backups '[] Response}.
        (The extra \inlinecode{'[]} denotes that no party yet knows all of the \inlinecode{oks}.)
        \inlinecode{gather} (line 8) communicates all the \inlinecode{oks} to the primary server
        where they're stored as a \inlinecode{Quire backups Response}.
        If all the backups are ok, then the primary server also handles the request (line 10).
    }
    \label{fig:census-poly-example}
    %\Description{23 lines of Haskell code using MultiChor defining a new version of "kvs" that is parametric over the number of backup servers.}
  \end{mdframed}
\end{figure*}

The first thing that is necessary is a way to loop over a polymorphic list of parties.
The second thing is the ability to express and use divergent data known by un-enumerated parties.
We call such data structures \emph{faceted values}\footnote{
    The word "faceted" is most commonly used in reference to a cut gemstone,
    but analogy to the facets of polyhedral playing dice might be more on-the-nose.
};
they're basically the same as the faceted values introduced in \cite{austin2012},
except their public facet is always "$\bot$" and multiple parties have distinct private facets.
Conceptually, a faceted value is similar to an MLV
(it projects to an owner as a simple value and to a non-owner as a placeholder),
but different owners of a faceted value will have different values for it.
To see the need for faceted values, consider how one would write a census-polymorphic \inlinecode{gather} operation
using only an appropriately-specialized \inlinecode{for}-loop:
Each sender would need to generate its value to send \emph{inside} the loop body,
and the only way for the sent values to be distinct would be by using private local state accessed by \inlinecode{locally}.
This would hardly be satisfying, and the dual case of \inlinecode{scatter} would be even worse:
Any use to which the received values were to be put would also have to fit inside the body of the \inlinecode{for}-loop.
(One couldn't simply append the \inlinecode{scatter}ed values to a list in a typed language
because \inlinecode{Located} values with different owners have different types.)
Our implementations provide these needed features in different ways as discussed in \Cref{sec:census-poly-haskell,sec:census-poly-rust}.

In all three of our implementations, census polymorphism can be resolved statically,
\ie, while one can write choreographies and choreographic functions that are census-polymorphic,
it is always possible in principal to unroll the top-level choreography
(that actually gets compiled)
into a monomorphic form.
(Therefore, we do not bother with a separate proof of the soundness of census polymorphism.)
In \MultiChor, this is denoted by \inlinecode{KnownSymbols} constraints in some type signatures.
We leave any clarity about the prospects of \emph{dynamic} census polymorphism for future work.


\subsection{Census Polymorphism in \MultiChor}
\label{sec:census-poly-haskell}

We leverage the type system of modern Haskell to achieve useful census polymorphism in \MultiChor.
This behavior is implemented as a layer \textit{on top of} \MultiChor's central monad and data-types;
from a theory perspective \MultiChor gets census polymorphism "for free" because it's a Haskell library.
The \MultiChor repository contains over a dozen example choreographies, several of which use census polymorphism.
In Figure~\ref{fig:census-poly-example} we showcase a key-value store choreography that's polymorphic over the number of backup servers.
\Cref{sec:mpc} presents a more involved census-polymorphic example.

Key to \MultiChor's strategy is Haskell's ability to express quantified type variables.
For example, a \inlinecode{Faceted} value is (underneath a little boiler-plate) a function.
Its argument is a \inlinecode{Member} proof that some party is in the list of owners,
and it returns a \inlinecode{Located} value known to the party in question.
Notably, nothing about the type, \inlinecode{Faceted ps cmn x}, indicates who the (type-level!) party indicated by the argument might be.
(The second type parameter, \inlinecode{cmn}, represents parties who know \emph{all} the contained values;
it's frequently \inlinecode{'[]}.)

\inlinecode{Faceted ps cmn x} is actually a special case of a more general type,
\inlinecode{PIndexed ps f}, where \inlinecode{f} can be any \emph{type-level function} from a party to a concrete type.
A \inlinecode{PIndexed} is effectively a type-indexed vector,
except that the type of the value retrieved depends on the index\footnote{
    The case where it does not depend on the index, \ie when \inlinecode{f} is \inlinecode{Const},
    is sufficiently useful in its own right that we package it as the type \inlinecode{Quire}.
    "Quire" is pronounced "choir"; it rhymes with "buyer" and means "a stack of sheets of paper, all cut to the same size".
    Each individual piece of paper is a "leaf".
}.
Because of its unusual \inlinecode{kind}, type classes that one would expect to apply to vectors generally do not apply to \inlinecode{PIndexed}.
What's actually needed for census polymorphism is the ability to \inlinecode{sequence} a \inlinecode{PIndexed} of choreographies.
Since \inlinecode{PIndexed} is not an instance of \inlinecode{Traversable},
we implement the needed function \inlinecode{sequenceP}, which is effectively just a \inlinecode{for}-loop
(in any monad) over type-level lists of parties.
These loops are not unrolled at compile time;
the type class \inlinecode{KnownSymbols} affords to the runtime environment sufficient knowledge of the type-level list.

\begin{figure*}[tbhp]
  \begin{mdframed}
    \inputminted[xleftmargin=10pt,linenos,fontsize=\scriptsize]{haskell}{figures/census-poly-haskell.hs.txt}
    \caption{
        Type signatures for \inlinecode{sequenceP}, \inlinecode{fanOut}, and \inlinecode{scatter}.
    }
    \label{fig:census-poly-haskell}
    %\Description{Fourteen lines of Haskell code using the MulitChor library.}
  \end{mdframed}
\end{figure*}

We could write a satisfactory \inlinecode{scatter} and \inlinecode{gather} directly using \inlinecode{sequenceP},
but we find that by defining the intermediary functions \inlinecode{fanOut} and \inlinecode{fanIn}
we can avoid almost all situations where a programmer might need to use \inlinecode{sequenceP} directly in a choreography.
\inlinecode{fanOut}'s argument is a choreography that results in a \inlinecode{Located} value at the party identified by the loop variable;
it aggregates these results as a \inlinecode{Faceted}.
\inlinecode{fanIn} is almost the same, except that the locations of the resulting values do not vary,
and they are aggregated in a \inlinecode{Quire} located at some list of recipients.
\Cref{fig:census-poly-haskell} shows the type signatures for \inlinecode{sequenceP}, \inlinecode{fanOut}, and \inlinecode{scatter}.
Keen readers may notice that the "\inlinecode{cmn}" parties' views of a \inlinecode{Faceted} are effectively just a \inlinecode{Quire},
and so wonder at the need for \inlinecode{fanIn}.
In fact, \inlinecode{fanIn} \emph{is} less often used than \inlinecode{fanOut},
but it's necessary for expressing choreographic loops that yield values which aren't known to the parties over whom the loop is defined.
For example, the GMW protocol, which we implement using \MultiChor in \Cref{sec:mpc}, cannot be written using only \inlinecode{fanOut}.

Modern Haskell language features, especially type-variable quantification,
enable \MultiChor's implementation of census polymorphism to be entirely type-safe and transparent to users.
This is a flexible system within which users can easily write their own novel and bespoke functions and data structures;
we will show in the following section that such flexibility is not entirely necessary for census polymorphism.


\section{Future Work}
\label{sec:future-implementation}
Roughly, summarize some of the open issues from github...
also maybe anything we're doing with Owen. That kinda thing.

\bibliographystyle{chicago}
\bibliography{refs}
