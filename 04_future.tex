\chapter{Beyond \MultiChor}
\label{sec:future}

We believe \MultiChor to be the best off-the-shelf system presently available for any real-world applications of CP.
We are also confident that \HLSCentral and the associated theorems demonstrate the theoretical soundness of the conclaves-\&-MLVs
CP paradigm.
That said, it is unsatisfying that the syntactic structures of these two systems are so different from each other.
Furthermore, it is not clear that these systems as they stand are good foundations for the development of more advanced CP techniques,
nor is it clear that the current design of the \MultiChor as a software library is ideal for real-world engineering.
In \Cref{sec:usability} we discuss some known usability problems with the existing \MultiChor implementation
(as distinct from theoretical limitations).
In \Cref{sec:future-implementation} we describe a fork of \MultiChor, a research prototype with some theoretical implications that will affect
the design of future \MultiChor versions.

\section{User challenges in \MultiChor}
\label{sec:usability}

Industry use of CP concepts remains nascent,
but enough embedded or semi-embedded implementations now exist or are in development that prospective users
will need to actively choose between them.
Just within the Haskell ecosystem, it's possible that an engineer might accept the excess communication necessitated by \HasChor's
KoC strategy in order to avoid the conceptual (and textual) overhead of census tracking.
Indeed, although we know of no "in the wild" use of \MultiChor,
anecdotal reports from academic peers who have attempted to use the library suggest it would benefit from substantial further cosmetic work,
and may need theoretical breakthroughs to appeal to non-academic developers.

\subsection{General Feedback}
\label{sec:usability-sources}
A few people that we know of (besides ourselves) have actually attempted to write programs using \MultiChor.
A couple of our fellow students accepted our invitation to do try a small programming exercise modeled after a job-interview exercise.
The exercise itself is described in \Cref{sec:usability-exercise}.
These sessions were not structured as a controlled usability study;
volunteers were invited to ask for help with any part of the exercise they wished.
Nonetheless, none of the volunteers were able to implement the described protocol,
which had been designed both for brevity and to exactly fit \MultiChor's capabilities.
This was our most detailed source of feedback.

While our own efforts to asses \MultiChor's usability were illuminating and negative,
we also received feedback on the software artifact~\cite{ourArtifact} submitted alongside~\cite{batesenclaves}.

One point of feedback has been practically unanimous:
\MultiChor's existing documentation is insufficient.
Relying on type signatures to communicate behavior presupposes familiarity with \MultiChor's types,
and the textual documentation, however systematic, is not suitable for bootstrapping a new user's understanding.
The example choreographies included in the \MultiChor repository are not presented as a form of documentation, 
and therefor do not serve that purpose.
Regarding specific directions for future documentation,
the documentation of \MultiChor's API should have relevant examples built into it,
and the entry-point of the documentation (the \inlinecode[bash]{README} file) should be structured as a tutorial.
Furthermore, the module structure of the repository should be refactored to reflect how the API will be \emph{used},
instead of how it was \emph{engineered}.

\subsection{The proof-witnesses system}

powerful, but not powerful enough and hard to use.

\subsection{"Compute this" operators}

There's too many of them.

\subsection{Clarity over flexibility}

get rid of CanSend.

\section{"Mini"-Chor}
\label{sec:future-implementation}
It is natural to ask why \MultiChor has a "core" API distinct from the ergonomic API afforded to end-users.
This design pattern makes reasoning about \MultiChor's implementation easier;
whether there are any performance implications has not been explored\footnote{
  The need for methods for comparing the performance of CP systems was acknowledged by the community
  of CP researchers attending PLDI24.
}.
In the coming months, we would like to push the simplicity of the core API even further:
\begin{enumerate}
  \item The use of a freer monad system is not actually necessary.
        It facilitates implementation, but it's also an additional "moving part" that can be removed
        without affecting the system's behavior.
\item As mentioned in \Cref{sec:multichor}, the functions \inlinecode{flatten} and \inlinecode{othersForget}
        could be removed from the core API and replaced with derived monadic functions.
        This would be a step backwards for the ergonomics of \MultiChor,
        so we do not advocate making such a change to the version published on Hackage.
  \item Replacing the core operation \inlinecode{congruently'} with \inlinecode{naked}
        (a monadic operator that unwraps an MLV known to the entire census)
        may degrade ergonomics.
        On the other hand, it would also allow simplification of \inlinecode{locally'}
        by obviating the \inlinecode{Unwrap} argument;
        \ie \inlinecode{locally'}'s argument would no longer be a function at all!
  \item Having made the above changes, the only remaining place where \inlinecode{Located} values would get unwrapped
        would be \inlinecode{naked}.
        We would be able to remove the underlying \inlinecode{Choreo} constructor for \inlinecode{naked}
        and change the implementation of \inlinecode{Located} to be a \inlinecode{newtype} wrapper for
        a census-polymorphic choreography yielding the target value.
        In other words, \inlinecode{naked} would be the field accessor function for \inlinecode{Located}.
\end{enumerate}

By making these changes, which we expect to preserve in a clearly labeled fork of \MultiChor,
we will facilitate subsequent theoretical work on CP.
In particular, the reduced API would make a good target for a more fully-featured formal model.

\bibliographystyle{chicago}
\bibliography{refs}
