\chapter{Ongoing Work}
\chaptermark{\MultiChor?} %this is the chapter heading that will show on subsequent pages
\label{sec:future}

We believe \MultiChor to be the best off-the-shelf system presently available for any real-world applications of CP.
We are also confident that \HLSCentral and the associated theorems demonstrate the theoretical soundness of the enclaves-\&-MLVs
CP paradigm.
That said, it is unsatisfying that the syntactic structures of these two systems are so different from each other.
Furthermore, it is not clear that these systems as they stand are good foundations for the development of more advanced CP techniques.
We propose to make changes to one or both systems in the coming months,
to make them align with each other better
and to facilitate subsequent work in this area.

\section{"Mini"-Chor}
\label{sec:future-implementation}
It is natural to ask why \MultiChor has a "core" API distinct from the ergonomic API afforded to end-users.
This design pattern makes reasoning about \MultiChor's implementation easier;
whether there are any performance implications has not been explored\footnote{
  The need for methods for comparing the performance of CP systems was acknowleged by the community
  of CP researchers attending PLDI24.
}.
In the coming months, we would like to push the simplicity of the core API even further:
\begin{enumerate}
  \item The use of a freer monad system is not actually necessary.
        It facilitates implementation, but it's also an additional "moving part" that can be removed
        without affecting the system's behavior.
\item As mentioned in \Cref{sec:multichor}, the functions \inlinecode{flatten} and \inlinecode{othersForget}
        could be removed from the core API and replaced with derived monadic functions.
        This would be a step backwards for the ergonomics of \MultiChor,
        so we do not advocate making such a change to the version published on Hackage.
  \item Replacing the core operation \inlinecode{congruently'} with \inlinecode{naked}
        (a monadic operator that unwraps an MLV known to the entire census)
        may degrade ergonomics.
        On the other hand, it would also allow simplification of \inlinecode{locally'}
        by obviating the \inlinecode{Unwrap} argument;
        \ie \inlinecode{locally'}'s argument would no longer be a function at all!
  \item Having made the above changes, the only remaining place where \inlinecode{Located} values would get unwrapped
        would be \inlinecode{naked}.
        We would be able to remove the underlying \inlinecode{Choreo} constructor for \inlinecode{naked}
        and change the implementation of \inlinecode{Located} to be a \inlinecode{newtype} wrapper for
        a census-polymorphic choreography yielding the target value.
        In other words, \inlinecode{naked} would be the field accessor function for \inlinecode{Located}.
\end{enumerate}

By making these changes, which we expect to preserve in a clearly labeled fork of \MultiChor,
we will facilitate subsequent theoretical work on CP.
In particular, the reduced API would make a good target for a more fully-featured formal model.

