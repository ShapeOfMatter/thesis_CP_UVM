\chapter{Beyond \MultiChor}
\label{sec:future}

We believe \MultiChor to be the best off-the-shelf system presently available for any real-world applications of CP.
We are also confident that \HLSCentral and the associated theorems demonstrate the theoretical soundness of the conclaves-\&-MLVs
CP paradigm.
That said, it is unsatisfying that the syntactic structures of these two systems are so different from each other.
Furthermore, it is not clear that these systems as they stand are good foundations for the development of more advanced CP techniques,
nor is it clear that the current design of the \MultiChor as a software library is ideal for real-world engineering.
In \Cref{sec:usability} we discuss some known usability problems with the existing \MultiChor implementation
(as distinct from theoretical limitations).
In \Cref{sec:future-implementation} we describe a fork of \MultiChor, a research prototype with some theoretical implications that will affect
the design of future \MultiChor versions.

\section{User challenges in \MultiChor}
\label{sec:usability}

Industry use of CP concepts remains nascent,
but enough embedded or semi-embedded implementations now exist or are in development that prospective users
will need to actively choose between them.
Just within the Haskell ecosystem, it's possible that an engineer might accept the excess communication necessitated by \HasChor's
KoC strategy in order to avoid the conceptual (and textual) overhead of census tracking.
Indeed, although we know of no "in the wild" use of \MultiChor,
anecdotal reports from academic peers who have attempted to use the library suggest it would benefit from substantial further cosmetic work,
and may need theoretical breakthroughs to appeal to non-academic developers.

\subsection{General Feedback}
\label{sec:usability-sources}
A few people that we know of (besides ourselves) have actually attempted to write programs using \MultiChor.
A couple of our fellow students accepted our invitation to do try a small programming exercise modeled after a job-interview exercise.
The exercise itself is described in \Cref{sec:usability-exercise}.
These sessions were not structured as a controlled usability study;
volunteers were invited to ask for help with any part of the exercise they wished.
Nonetheless, none of the volunteers were able to implement the described protocol,
which had been designed both for brevity and to exactly fit \MultiChor's capabilities.
This was our most detailed source of feedback.

While our own efforts to asses \MultiChor's usability were illuminating and negative,
we also received feedback on the software artifact~\cite{ourArtifact} submitted alongside~\cite{batesenclaves}.

One point of feedback has been practically unanimous:
\MultiChor's existing documentation is insufficient.
Relying on type signatures to communicate behavior presupposes familiarity with \MultiChor's types,
and the textual documentation, however systematic, is not suitable for bootstrapping a new user's understanding.
The example choreographies included in the \MultiChor repository are not presented as a form of documentation, 
and therefor do not serve that purpose.
Regarding specific directions for future documentation,
the documentation of \MultiChor's API should have relevant examples built into it,
and the entry-point of the documentation (the \inlinecode[bash]{README} file) should be structured as a tutorial.
Furthermore, the module structure of the repository should be refactored to reflect how the API will be \emph{used},
instead of how it was \emph{engineered}.

\subsection{The proof-witnesses system}

A major hurdle to writing correct choreographies with \MultiChor is constructing and managing the proof-witness arguments.
Regardless of whether or not the system is overly-complicated
(to quote both the volunteers from the usability exercise:
\emph{"It's kinda complicated."})
the cognitive load of using the proof-witness system is \emph{additional} to, and mostly \emph{perpendicular} to,
the complexity of writing the actual choreography.
In other words,
a user must consider both how to represent a choreographic behavior using \MultiChor's operators
and how to prove that the relevant parties have the relevant memberships,
and because the proofs serve double-duty as identifiers, the user must think about those two problems simultaneously.

It's not unusual for a Haskell library to have a steep learning curve,
but we observe that the proof-witness system is a real bottleneck to use of \MultiChor,
so it would be worth improving.
Furthermore, the existing system is lacking certain capabilities one would expect,
such as the ability to talk about set unions, intersections, and differences.
There are two plausible directions for improvident;
neither of them are perfectly clear at this time:
\begin{enumerate}
	\item Remove the witnesses altogether in favor of constraint programming.
		(\inlinecode{Proxy} objects would still be necessary in some cases, but contain no complexity.)
		As explained in \Cref{sec:membership}, normal \inlinecode{class}es will not suffice,
		but quantified constraints~\cite{bottu2017} are not yet ruled out.
		Specifically, type-level programming with GHC's \inlinecode{QuantifiedConstraints} extension
		will be able to automatically infer sufficient subset relations,
		\emph{if} a satisfactory representation of membership constraints can be expressed.
	\item Externalize the proof-witness system.
		Several experimental systems exist for reasoning about type-level sets in Haskell,
		including \inlinecode{gdp}~\cite{gdp_hackage} and \inlinecode{type-level-sets}~\cite{tls_hackage}.
		If truly no off-the-self library exists that's suitable for the challenges of \MultiChor,
		then \MultiChor's solution should be separated as a stand-alone library.
		Regardless of where an external library came from, its externalness would facilitate code-reuse 
		and clarify a separation of concerns.
\end{enumerate}

\subsection{"Compute this" operators}

\MultiChor offers three "basic" operators for embedding non-choreographic computation in a choreography:
\inlinecode{locally}, \inlinecode{congruently}, and \inlinecode{parallel}.
Each of these is derived from more primitive forms, and each has further derived forms
(\eg \inlinecode{_locally_}, \inlinecode{purely}).
Choosing the best of these options for any given task is worse than a needle-in-haystack problem,
because multiple of them may actually work,
and because some may appear to work for the immediate task while causing problems later in the program.

In \Cref{sec:future-implementation} we will see a possibility for removing some of this complexity.
Regardless of whether the development of \MultiChor actually goes in that direction,
the cost (as measured in end-user boilerplate) of removing (or not exporting) most of these functions
would probably be worth the benefit of simplifying the choices they present to end-users.

\subsection{Clarity over flexibility}

To send a value in \MultiChor one must (in addition to specifying the recipients)
provide a MLV and prove that the sender both owns the value and is present in the census.
This can be quite repetitive.
To minimise boiler-plate, the surface API of \MultiChor uses a class \inlinecode{CanSend}
so that the broad- and multi-cast functions can take the proof arguments in different formats.
In keeping with the above theme of reducing the space of options users must navigate,
\inlinecode{CanSend} should be removed and the functions should each have a single, general-purpose, signature.

More broadly, we suggest abandoning the existing implementation-focused module structure
that separates operators across three modules ("core", "surface", and "batteries")
in favor of exporting a single surface-level API.
We expect that users will build predictable helper functions on top of this,
but trying to preempt their efforts doesn't seem to have helped prospective users.

\section{"Mini"-Chor}
\label{sec:future-implementation}

Entirely perpendicular to questions of ergonomics and learning-curves, the expressivity of \MultiChor could be improved
(\eg failable communication),
and the \HLSCentral system is not readily adaptable for proving the safety of further extensions.
Therefore, in this section we discuss a fork of \MultiChor, \minichor,
which is able to express all the same choreographies (with caveats noted in \Cref{sec:absolute-owners})
using a parred-down core API which we believe is simple enough to directly model in a formalism.
\todo{cite minichor}
We do not present such a model at this time.
We also don't present \minichor as "\MultiChor-V2",
because it's differences from \MultiChor are probably negative if measured in terms of performance or ergonomics.

Most of this section will describe the differences between \MultiChor and \minichor narratively.
The first change is to remove the freer-monad system and instead implement \inlinecode{Functor}, \inlinecode{Applicative}, and \inlinecode{Monad}
for \inlinecode{Choreo} directly.
This has no effect on the rest of the system or on the case studies;
it's simply a moving part which we have the ability to remove\footnote{
	The main selling point of freer monads is how they compose with each other,
	and how little boiler-plate is needed when writing them.
	Neither of these are needed for \minichor.
	There may also be performance considerations;
	the need for methods for comparing the performance of CP systems was acknowledged by the community
  of CP researchers attending PLDI24.
	}.
Second, we remove \inlinecode{othersForget} and \inlinecode{flatten} from the core API and re-implement them as
monadic operations in the surface API using \inlinecode{congruently'}.
This requires some small changes in the case studies; \eg
\begin{minted}[xleftmargin=30pt,fontsize=\small]{haskell}
do result <- (listedFirst,
              alice @@ nobody,
              flatten aliceInConclaveA aliceinConclaveB value) ~> bob @@ nobody
   return result
\end{minted}
becomes
\begin{minted}[xleftmargin=30pt,fontsize=\small]{haskell}
do value' <- flatten aliceInConclaveA aliceinConclaveB value
   result <- (listedFirst, alice @@ nobody, value') ~> bob @@ nobody
   return result
\end{minted}
Third, we remove the type parameter \inlinecode{m} (for monad) from \inlinecode{Choreo}
and simply assume that the local monad is always \inlinecode{CLI IO}.
This is basically the same as just \inlinecode{IO}, and use-cases for local monads that \emph{aren't} basically just \inlinecode{IO}
seem uncommon.
The rest of the changes descried in this section are more impactful,
but the process follows a similar pattern of refactoring the core API and then either shimming the difference in the surface API
(so that the exposed system behaves the same)
or propagating semantically-inconsequential changes into the case studies.
Most of the case studies have robust unit tests based on them, to detect any mistakes during this process.

\subsection{Monadic Unwrapping}
\HasChor enforces the rule that only the owner of a located value may call \inlinecode{unwrap} on it
by hiding \inlinecode{unwrap} in a module (only its type, \inlinecode{Unwrap} is exported)
and affording it to users only as an argument to \inlinecode{locally}'s callback argument.
\MultiChor uses the exact same pattern, but a design goal was to also represent \emph{pure} computation
actively replicated across the owners of the relevant MLVs.
The way \MultiChor does this is by duplicating the \inlinecode{locally}\inlinecode{Unwrap} pattern
to make \inlinecode{congruently}\inlinecode{Unwraps},
as shown in \Cref{fig:minichor-stg1}(a).

An alternative to \inlinecode{congruently'}
(which actively replicates a pure computation using MLVs known to the entire census)
is \inlinecode{naked},
which unwraps a single MLV known to the entire census.
The two strategies are equivalent in what they can express,
but \inlinecode{naked} has the disadvantage that it can't be adapted for use in a larger census as ergonomically as \inlinecode{congruently'} can;
the equivalent of the un-primed \inlinecode{congruently} in a \inlinecode{naked}-based system is a family of
functions for each fixed $N$ that each handle pure computations on $N$ arguments.
Since \minichor doesn't care about ergonomics, this is acceptable.

The advantage of replacing \inlinecode{congruently'} with \inlinecode{naked} is that it can also replace the call-back pattern of \inlinecode{locally}.
This intermediate system is shown in \Cref{fig:minichor-stg1}(b).

\begin{figure*}[tbhp]
  \begin{mdframed}
    \begin{tabular}{r l}
	    \begin{minipage}{1cm}
	    \textbf{(a)} 
	    \end{minipage}&
	    \begin{minipage}{11cm}
	    \inputminted[xleftmargin=10pt,linenos,fontsize=\scriptsize,firstnumber=1,firstline=1,lastline=9]{haskell}{figures/minichor_stg1.hs.txt}
	    \end{minipage}\\[6em]
	    \begin{minipage}{1cm}
	    \textbf{(b)}
	    \end{minipage}&
	    \begin{minipage}{11cm}
	    \inputminted[xleftmargin=10pt,linenos,fontsize=\scriptsize,firstnumber=1,firstline=15,lastline=25]{haskell}{figures/minichor_stg1.hs.txt}
	    \end{minipage}
    \end{tabular}
    \caption{
	    Different strategies for local effects and pure active replication.
	    \textbf{(a)} The \MultiChor approach. The two types \inlinecode{Unwrap} and \inlinecode{Unwraps} are used as the argument types in callback functions used by \inlinecode{locally'} and \inlinecode{congruently'}.
	    \inlinecode{naked} in this system is a derived function.
	    \textbf{(b)} The \inlinecode{naked}-based approach.
	    In this system, \inlinecode{locally'} just lifts local monadic effects (\inlinecode{CLI IO}) into singleton choreographies
	    (which can be conclaved).
	    The pseudo-code \inlinecode{congruentlyN} shows how,
	    for any fixed number $N$ of MLVs that will be used in the pure computation,
	    an analog of \inlinecode{congruently} can be written.
	    A similar pattern for \inlinecode{locallyN} is not shown.
    }
    \label{fig:minichor-stg1}
  \end{mdframed}
\end{figure*}

\begin{figure*}[tbhp]
  \begin{mdframed}
    \begin{tabular}{r l}
	    \begin{minipage}{1cm}
	    \textbf{(a)} 
	    \end{minipage}&
	    \begin{minipage}{11cm}
	    \inputminted[xleftmargin=10pt,linenos,fontsize=\scriptsize,firstnumber=1,firstline=1,lastline=16]{haskell}{figures/minichor_stg2.hs.txt}
	    \end{minipage}\\[7em]
	    \begin{minipage}{1cm}
	    \textbf{(b)}
	    \end{minipage}&
	    \begin{minipage}{11cm}
	    \inputminted[xleftmargin=10pt,linenos,fontsize=\scriptsize,firstnumber=1,firstline=20,lastline=36]{haskell}{figures/minichor_stg2.hs.txt}
	    \end{minipage}\\[7em]
	    \begin{minipage}{1cm}
	    \textbf{(c)}
	    \end{minipage}&
	    \begin{minipage}{11cm}
	    \inputminted[xleftmargin=10pt,linenos,fontsize=\scriptsize,firstnumber=1,firstline=40,lastline=56]{haskell}{figures/minichor_stg2.hs.txt}
	    \end{minipage}
    \end{tabular}
    \caption{
	    Under-the-hood implementation changes for redefining MLVs out of existence.
	    \textbf{(a)} shows the \inlinecode{naked}-based system from \Cref{fig:minichor-stg1}.
	    \textbf{(b)} shows another intermediate system described in \Cref{sec:minichor-stg2}.
	    \textbf{(c)} shows \minichor.
	    Each block lists the core operations of the \inlinecode{Choreo} monad (lines~4, 6, and~4--6 respectively),
	    the representation of \inlinecode{Located} values, and the interesting cases of the EPP function.
	    The data constructors \inlinecode{Wrap} and \inlinecode{Empty} (\textbf{(a)}~line~1)
	    and the AST form \inlinecode{Naked} (line~6), are absent in \textbf{(b)}.
	    Instead, \inlinecode{naked} is the accessor of the data type \inlinecode{Located},
	    which wraps a function from proof of ownership to a choreography over the specified subset of the owners (\textbf{(b)}~line~2).
	    At runtime, the placeholder used for remote MLVs is a choreography that returns \inlinecode{undefined} (an error) (line~16).
	    In practice one's own MLVs will be represented at runtime by ASTs for trivial choreographies (\eg \inlinecode{Return 5}),
	    this is what \inlinecode{pure} does in the \inlinecode{Choreo} monad (\textbf{(b)}~line~15).
	    In \textbf{(c)}, \inlinecode{Located} is just an alias for \inlinecode{Choreo}.
	    As discussed in \Cref{sec:minichor-stg3}, this requires swapping \inlinecode{conclave} for \inlinecode{conclaveTo}
	    and changing the signature of \inlinecode{broadcast'}.
	    The implementation of EPP is basically the same, there's just no construction or unwrapping of located values;
	    \inlinecode{naked} no longer exists.
    }
    \label{fig:minichor-stg2}
  \end{mdframed}
\end{figure*}


\subsection{MLVs as quantified functions}
\label{sec:minichor-stg2}
In the \inlinecode{naked}-based system of \Cref{fig:minichor-stg1}(b),
\inlinecode{naked} is the only means by which the actual value of an MLV can be accessed.
This suggests removing \inlinecode{naked} from the foundational signature of \inlinecode{Choreo},
and instead making it the actual definition of \inlinecode{Located}.
\Cref{fig:minichor-stg2}\textbf{(b)} shows this change.

A design pattern of \MultiChor was that the \inlinecode{Core} module needed to be "trusted";
our own reasoning outside of Haskell's type system is what guarantees that no user working outside of \inlinecode{Core}
can call \inlinecode{unwrap} on \inlinecode{Empty}.
None of our changes in the \minichor fork alter this pattern;
even the nature of the invariant we're maintaining is the same:
That a party will never compute on an MLV they don't own.
The change in \Cref{fig:minichor-stg2} is just where the impossible error lives,
from the case-wise definition of \inlinecode{unwrap} to an undefined value returned by a choreography generated at runtime
(\Cref{fig:minichor-stg2}\textbf{(b)}~line~16).

\subsection{MLVs \emph{are} Choreographies}
\label{sec:minichor-stg3}
The change described in \Cref{sec:minichor-stg2} has almost no effect on the exposed API; it just swaps the order of \inlinecode{naked}'s arguments.
At the same time, it may be unclear what the point of it was.
The point of it was to make intuitive the remaining jumps to \minichor, a core API for choreographic programming that
doesn't have located values at all!

We get rid of MLVs by relaxing our demands of them.
Previously it sufficed for one or more owners of an MLV to be present in a census for them to do something with that value,
but now we will require that \emph{all} owners be present.
In terms of implementation,
we demote \inlinecode{Located} from a \inlinecode{newtype} wrapper around a function down to just a type alias for \inlinecode{Choreo}
(\Cref{fig:minichor-stg2}\textbf{(b)}).
To understand the conceptual difference,
consider some formal DSL of no specific purpose:
the syntax of expressions in that language contains as a subset it's syntax of values.
In other words, $5$ is a computation that happens to evaluate to the same thing as $2+3$.
Similarly, in our earlier model \HLSCentral, $5@\nonempty{p}$ is a computation that evaluates by $\nonempty{p}$ to five;
we promise that no-one not in $\nonempty{p}$ will attempt to evaluate it, and such non-owning parties replace it with $\bot$ at runtime.
Any (multiply) located value like $5@\nonempty{p}$ can be perfectly represented by a choreography which
\begin{itemize}
	\item has exactly $\nonempty{p}$ as its census and
	\item evaluates to the (not located) value "five".
\end{itemize}

Giving up the ability to use a "located value" when not all of its owners are present has two big effects on the overall system.
First, reusable software components can no longer take arguments with open-ended polymorphic ownership sets;
an MLV is useless without proof that all its owners are present in the choreography.
\inlinecode{othersForget} can still be used to reduce ownership sets, but it now needs all the original owners to also be present.
Often, it's necessary to apply \inlinecode{othersForget} one or more layers up in the program's architecture from where the value gets used,
an reusable components should be strict instead of lenient about the owners of their arguments.

A more fundamental change is that it is no longer possible to write the function \inlinecode{flatten}.
Consider its hypothetical type signature:
\begin{minted}[xleftmargin=30pt,fontsize=\small]{haskell}
flatten :: (KnownSymbols ls) =>
  Subset newOwners census -> Subset newOwners outer -> Subset newOwners inner ->
  Located outer (Located inner a) -> Choreo census (Located newOwners a)
\end{minted}
An implementation would take as an argument a \inlinecode{Located outer (...)};
in order to \emph{use} that it would have to conclave to \inlinecode{outer}.
Inside the conclave, it would have a \inlinecode{Located inner a},
but there'd be nothing it could do with it because there'd be no proof that all of \inlinecode{inner} are present in \inlinecode{outer}.
There may be multiple solutions to this problem; \minichor's solution is to make \inlinecode{flatten} unnecessary
by replacing the core operation \inlinecode{conclave} with \inlinecode{conclaveTo},
who's body-argument is required to return a located value, and which does not add a layer of location-wrapping
(\Cref{fig:minichor-stg2}\textbf{(c)}~line~6).
(In \MultiChor, \inlinecode{conclaveTo} is a derived function using \inlinecode{flatten}.
In \minichor, \inlinecode{conclave} is a derived function using monad-bind.)


\subsection{Implications}
1: we have a good model.
2: this clearly dovetails with lazy expressions.
3: MLVs are functors!


\bibliographystyle{chicago}
\bibliography{refs}
