\chapter{\minichor is \MultiChor, Just Smaller}
\label{sec:future}

\HLSCentral and the associated theorems demonstrate the theoretical soundness of the conclaves-\&-MLVs
CP paradigm,
and \MultiChor demonstrates that it is practical to implement and use that paradigm.
That said, it is unsatisfying that the syntactic structures of these two systems are so different from each other.
Furthermore, it is not clear that these systems as they stand are good foundations for the development of more advanced CP techniques:
the \HLSCentral system is not readily adaptable for proving the safety of further extensions,
and \MultiChor has some known usability problems as discussed in \Cref{sec:usability}.
As a step toward future practical and theoretical developments, we developed \minichor~\cite{minichor-repo},
a research-prototype fork of \MultiChor.
\minichor is able to express all the same choreographies (see caveat in \Cref{sec:minichor-stg3})
using a parred-down core API which we believe is simple enough to directly model in a formalism.
\minichor also demonstrates some immediate insights that will affect
the design of future \MultiChor versions.

Of particular note is that \minichor does not feature located values (multiply or singly) as understood by prior CP systems.
While the type \inlinecode{Located} appears in \minichor and is used similarly to the type of the same name in \MultiChor,
in \minichor \inlinecode{Located} is just an alias for \inlinecode{Choreo}!
The theoretical implication of this is that MLVs are actually just a special case of census-annotated choregraphies.
We discuss further implications in \Cref{sec:minichor-implications}.

Most of this chapter will describe the differences between \MultiChor and \minichor narratively.
The first change is to remove the freer-monad system and instead implement \inlinecode{Functor}, \inlinecode{Applicative}, and \inlinecode{Monad}
for \inlinecode{Choreo} directly.
This has no effect on the rest of the system or on the case studies;
it's simply a moving part which we have the ability to remove\footnote{
	The main selling point of freer monads is how they compose with each other,
	and how little boiler-plate is needed when writing them.
	Neither of these are needed for \minichor.
	There may also be performance considerations;
	the need for methods for comparing the performance of CP systems was acknowledged by the community
  of CP researchers attending PLDI24.
	}.
Second, we remove \inlinecode{othersForget} and \inlinecode{flatten} from the core API and re-implement them as
monadic operations in the surface API using \inlinecode{congruently'}.
This requires some small changes in the case studies; \eg
\begin{minted}[xleftmargin=30pt,fontsize=\small]{haskell}
do result <- (listedFirst,
              alice @@ nobody,
              flatten aliceInConclaveA aliceinConclaveB value) ~> bob @@ nobody
   return result
\end{minted}
becomes
\begin{minted}[xleftmargin=30pt,fontsize=\small]{haskell}
do value' <- flatten aliceInConclaveA aliceinConclaveB value
   result <- (listedFirst, alice @@ nobody, value') ~> bob @@ nobody
   return result
\end{minted}
Third, we remove the type parameter \inlinecode{m} (for monad) from \inlinecode{Choreo}
and simply assume that the local monad is always \inlinecode{CLI IO}.
This is basically the same as just \inlinecode{IO}, and use-cases for local monads that \emph{aren't} basically just \inlinecode{IO}
seem uncommon.
The rest of the changes descried in this chapter are more impactful,
but the process follows a similar pattern of refactoring the core API and then either shimming the difference in the surface API
(so that the exposed system behaves the same)
or propagating semantically-inconsequential changes into the case studies.
Most of the case studies have robust unit tests based on them, to detect any mistakes during this process.

\section{Monadic Unwrapping}
\label{sec:minichor-stg1}
We replace the core operator \inlinecode{congruently'} with \inlinecode{naked}, simplifying the core API.

\HasChor enforces the rule that only the owner of a located value may call \inlinecode{unwrap} on it
by hiding \inlinecode{unwrap} in a module (only its type, \inlinecode{Unwrap} is exported)
and affording it to users only as an argument to \inlinecode{locally}'s callback argument.
\MultiChor uses the exact same pattern, but a design goal was to also represent \emph{pure} computation
actively replicated across the owners of the relevant MLVs.
The way \MultiChor does this is by duplicating the \inlinecode{locally}\inlinecode{Unwrap} pattern
to make \inlinecode{congruently}\inlinecode{Unwraps},
as shown in \Cref{fig:minichor-stg1}\textbf{(a)}.

An alternative to \inlinecode{congruently'}
(which actively replicates a pure computation using MLVs known to the entire census)
is \inlinecode{naked},
which unwraps a single MLV known to the entire census.
The two strategies are equivalent in what they can express,
but \inlinecode{naked} has the disadvantage that it can't be adapted for use in a larger census as ergonomically as \inlinecode{congruently'} can;
the equivalent of the un-primed \inlinecode{congruently} in a \inlinecode{naked}-based system is a family of
functions for each fixed $N$ that each handle pure computations on $N$ arguments.
Since \minichor doesn't care about ergonomics, this is acceptable.

The advantage of replacing \inlinecode{congruently'} with \inlinecode{naked} is that it can also replace the call-back pattern of \inlinecode{locally}.
This intermediate system is shown in \Cref{fig:minichor-stg1}\textbf{(b)}.

\begin{figure*}[tbhp]
  \begin{mdframed}
  \begin{tabular}{c}
  \begin{minipage}{0.95\linewidth}
	    \inputminted[xleftmargin=10pt,linenos,fontsize=\footnotesize,firstnumber=1,firstline=1,lastline=9]{haskell}{figures/minichor_stg1.hs.txt}
  \end{minipage} \\\\
  \begin{minipage}{0.95\linewidth}
	    \textbf{(a)} The \MultiChor approach. The two types \inlinecode{Unwrap} and \inlinecode{Unwraps} are used as the argument types in callback functions used by \inlinecode{locally'} and \inlinecode{congruently'}.
	    \inlinecode{naked} in this system is a derived function.
  \end{minipage} \\\\
  \hline\\
  \begin{minipage}{0.95\linewidth}
	    \inputminted[xleftmargin=10pt,linenos,fontsize=\footnotesize,firstnumber=1,firstline=15,lastline=25]{haskell}{figures/minichor_stg1.hs.txt}
  \end{minipage} \\\\
  \begin{minipage}{0.95\linewidth}
	  \textbf{(b)} The \inlinecode{naked}-based approach described in \Cref{sec:minichor-stg1}.
	    In this system, \inlinecode{locally'} just lifts local monadic effects (\inlinecode{CLI IO}) into singleton choreographies
	    (which can be conclaved).
	    The pseudo-code \inlinecode{congruentlyN} shows how,
	    for any fixed number $N$ of MLVs that will be used in the pure computation,
	    an analog of \inlinecode{congruently} can be written.
	    A similar pattern for \inlinecode{locallyN} is not shown.
  \end{minipage}
  \end{tabular}
    \caption{
	    Different strategies for local effects and pure active replication.
    }
    \label{fig:minichor-stg1}
  \end{mdframed}
\end{figure*}

\begin{figure*}[tbhp]
  \begin{mdframed}
  \begin{tabular}{c}
  \begin{minipage}{0.95\linewidth}
	    \inputminted[xleftmargin=10pt,linenos,fontsize=\footnotesize,firstnumber=1,firstline=1,lastline=18]{haskell}{figures/minichor_stg2.hs.txt}
  \end{minipage} \\\\
  \begin{minipage}{0.95\linewidth}
	    \textbf{(a)}
	    The \inlinecode{naked}-based system from \Cref{sec:minichor-stg1} and \Cref{fig:minichor-stg1}\textbf{(b)}.
  \end{minipage} \\\\
  \hline\\
  \begin{minipage}{0.95\linewidth}
	    \inputminted[xleftmargin=10pt,linenos,fontsize=\footnotesize,firstnumber=1,firstline=22,lastline=40]{haskell}{figures/minichor_stg2.hs.txt}
  \end{minipage} \\\\
  \begin{minipage}{0.95\linewidth}
	    \textbf{(b)}
	    Another intermediate system described in \Cref{sec:minichor-stg2}.
	    The data constructors \inlinecode{Wrap} and \inlinecode{Empty} (\textbf{(a)}~line~1)
	    and the AST form \inlinecode{Naked} (\textbf{(a)}~line~6), are absent.
	    Instead, \inlinecode{naked} is the accessor of the data type \inlinecode{Located},
	    which wraps a function from proof of ownership to a choreography over the specified subset of the owners (line~2).
	    At runtime, the placeholder used for remote MLVs is a choreography that returns \inlinecode{undefined} (an error) (line~16).
	    In practice one's own MLVs will be represented at runtime by ASTs for trivial choreographies (\eg \inlinecode{Return 5}),
	    this is what \inlinecode{pure} does in the \inlinecode{Choreo} monad (line~15).
  \end{minipage}
  \end{tabular}
    \caption{
	    Under-the-hood implementation changes for redefining MLVs out of existence.
	    \textit{(1/2)}
    }
    \label{fig:minichor-stg2ab}
  \end{mdframed}
\end{figure*}

\begin{figure*}[tbhp]
  \begin{mdframed}
  \begin{tabular}{c}
  \begin{minipage}{0.95\linewidth}
	    \inputminted[xleftmargin=10pt,linenos,fontsize=\footnotesize,firstnumber=1,firstline=44,lastline=63]{haskell}{figures/minichor_stg2.hs.txt}
  \end{minipage} \\\\
  \begin{minipage}{0.95\linewidth}
	  The \minichor system described in \Cref{sec:minichor-stg3}.
	    Here \inlinecode{Located} is just an alias for \inlinecode{Choreo}.
	    As discussed in \Cref{sec:minichor-stg3}, this requires swapping \inlinecode{conclave} for \inlinecode{conclaveTo}
	    and changing the signature of \inlinecode{broadcast'}.
	    The implementation of EPP is basically the same, there's just no construction or unwrapping of located values;
	    \inlinecode{naked} no longer exists.
  \end{minipage}
  \end{tabular}
    \caption{
	    Under-the-hood implementation changes for redefining MLVs out of existence.
	    \textit{(2/2)}
    }
    \label{fig:minichor-stg2c}
  \end{mdframed}
\end{figure*}


\section{MLVs as quantified functions}
\label{sec:minichor-stg2}
In the \inlinecode{naked}-based system of \Cref{fig:minichor-stg1}\textbf{(b)},
\inlinecode{naked} is the only means by which the actual value of an MLV can be accessed.
This suggests removing \inlinecode{naked} from the foundational signature of \inlinecode{Choreo},
and instead making it the actual definition of \inlinecode{Located}.
\Cref{fig:minichor-stg2ab}\textbf{(b)} shows this change.

A design pattern of \MultiChor was that the \inlinecode{Core} module needed to be "trusted";
our own reasoning outside of Haskell's type system is what guarantees that no user working outside of \inlinecode{Core}
can call \inlinecode{unwrap} on \inlinecode{Empty}.
None of our changes in the \minichor fork alter this pattern;
even the nature of the invariant we're maintaining is the same:
That a party will never compute on an MLV they don't own.
The changes in \Cref{fig:minichor-stg2ab,fig:minichor-stg2c} are only to where the impossible error lives,
from the case-wise definition of \inlinecode{unwrap} to an undefined value returned by a choreography generated at runtime
(\Cref{fig:minichor-stg2ab}\textbf{(b)}~line~16).

The point of the change described in \Cref{fig:minichor-stg2ab}\textbf{(b)} is to make intuitive the remaining jumps to \minichor,
a core API for choreographic programming that
doesn't have located values at all!

\section{MLVs \emph{are} Choreographies}
\label{sec:minichor-stg3}

We get rid of MLVs by relaxing our demands of them.
Previously it sufficed for one or more owners of an MLV to be present in a census for them to do something with that value,
but now we will require that \emph{all} owners be present.
In terms of implementation,
we demote \inlinecode{Located} from a \inlinecode{newtype} wrapper around a function down to just a type alias for \inlinecode{Choreo}
(\Cref{fig:minichor-stg2c}~line~1).

To understand the conceptual difference,
consider some formal DSL of no specific purpose:
the syntax of expressions in that language contains as a subset its syntax of values.
In other words, $5$ is a computation that happens to evaluate to the same thing as $2+3$.
Similarly, in our earlier model \HLSCentral, $5@\nonempty{p}$ is a computation that evaluates by $\nonempty{p}$ to five;
we promise that no-one not in $\nonempty{p}$ will attempt to evaluate it, and such non-owning parties replace it with $\bot$ at runtime.
Any (multiply) located value like $5@\nonempty{p}$ can be perfectly represented by a choreography which
\begin{itemize}
	\item has exactly $\nonempty{p}$ as its census and
	\item evaluates to the (not located) value "five".
\end{itemize}

Giving up the ability to use a "located value" when not all of its owners are present has two big effects on the overall system.
First, reusable software components can no longer take arguments with open-ended polymorphic ownership sets;
an MLV is useless without proof that all its owners are present in the choreography.
\inlinecode{othersForget} can still be used to reduce ownership sets, but it now needs all the original owners to also be present.
This implies a small \textbf{caveat} on the assertion that \minichor can do anything \MultiChor can do:
Higher-order choreographies in \minichor enjoy slightly less-open-ended polymorphic typing than in \MultiChor.
Reusable components must be restrictive (instead of lenient) about the owners of their arguments.
In the case studies, we were still able to preserve all the same behavior
by applying \inlinecode{othersForget} one or more layers up in the program's architecture from where the values in question are actually used.

A more fundamental change is that it is no longer possible to write the function \inlinecode{flatten}.
Consider its hypothetical type signature:
\begin{minted}[xleftmargin=30pt,fontsize=\small]{haskell}
flatten :: (KnownSymbols ls) =>
  Subset newOwners census -> Subset newOwners outer -> Subset newOwners inner ->
  Located outer (Located inner a) -> Choreo census (Located newOwners a)
\end{minted}
An implementation would take as an argument a \inlinecode{Located outer (...)};
in order to \emph{use} that it would have to conclave to \inlinecode{outer}.
Inside the conclave, it would have a \inlinecode{Located inner a},
but there'd be nothing it could do with it because there'd be no proof that all of \inlinecode{inner} are present in \inlinecode{outer}.
There may be multiple solutions to this problem; \minichor's solution is to make \inlinecode{flatten} unnecessary
by replacing the core operation \inlinecode{conclave} with \inlinecode{conclaveTo},
who's body-argument is required to return a located value, and which does not add a layer of location-wrapping
(\Cref{fig:minichor-stg2c}~line~6).
(In \MultiChor, \inlinecode{conclaveTo} is a derived function using \inlinecode{flatten}.
In \minichor, \inlinecode{conclave} is a derived function using monad-bind.)


\section{Implications}
\label{sec:minichor-implications}
Although \minichor was not intended to ever see real-world use,
the usability trade-offs between it and \MultiChor are not obvious,
and we can learn several things from it which can guide the development of a future CP frameworks.
Most significantly, MLVs are emergent rather than fundamental to choreographic programming!

Many prior CP systems (\eg Pirouette~\cite{hirsch2021pirouette}) have featured syntactic distinctions between "global" choreographic code
and local code;
these systems feature dualities between choreographic variables and local variables (which have their own namespaces)
and choreographic and local functions (which have distinct syntax).
In contrast, in \minichor choreographies are first-class values in the same syntax and namespace as all other Haskell code;
one can (and does) write values with types like \inlinecode{Choreo ps (Choreo qs Int)}.
Other prior systems (\eg \chorLambda~\cite{chor-lambda}) keep everything in a single "layer",
all values are located and all computation is choreographic.
Such systems still maintain distinct senses in which a value can be located:
data is annotated with owners and functions are annotated with participants;
these annotations are handled differently by the typing and semantic rules.
Because \minichor privileges the un-located ("naked") level of computation and treats choreographies as first-class values,
it is able to use the single syntactic construct \inlinecode{conclaveTo} (and the corresponding typing property, censuses)
as the sole ascription of locality.

That located values are (or can be) an emergent pattern in choreographic programming is a significant theoretical insight on its own.
Whether or not this pattern can be replicated for other styles of CP (\eg select-\&-merge) is beyond the scope of this work.
Here we explore three ways the concepts used in \minichor may affect the future development of \MultiChor.

First, as was the original intention, \minichor is a minimalist implementation;
sufficiently concise in its inner workings to be targeted by a formal model.
The \inlinecode{Choreo} data type (the ASTs of choreographies) has three important constructors,
plus the \inlinecode{Return} and \inlinecode{Bind} constructors it needs to implement \inlinecode{Functor} and \inlinecode{Monad}.
Each of the constructors individually is simple;
excepting the proof witnesses they each take a single argument.
Two concise functions \inlinecode{epp} and \inlinecode{runChoreo} implement the distributed and centralized semantics, respectively.
We leave as future work to compose a formal model of \minichor,
and to prove theorems about it
(especially that it enjoys some equivalence with a corresponding select-\&-merge system).
 
Second, while the actual implementation of \minichor doesn't use any laziness beyond what's normal for Haskell programs,
the use of \emph{choreographies} as the arguments and return-types of the choreographic operators
makes laziness-based programming patterns immediately available to users.
Specifically, while the MLVs that arise naturally in naïve use of \minichor are generally trivial \inlinecode{Return} ASTs,
nothing about the type system forbids passing a non-trivial choreography into any function that consumes an MLV.
One can write a \inlinecode{lazyMulticast x} that performs no immediate action itself but just returns
(as the "MLV") the choreography \inlinecode{broadcast x}.
It would be worthwhile to explore the utility, performance, and limitations of such a system,
and consider recapitulating it in \MultiChor.

Third, MLVs are functors!
(Specifically, they are endofunctors, which is what's meant by the Haskell type class \inlinecode{Functor}.
They are also instances of \inlinecode{Applicative} and \inlinecode{Monad}.)
This insight can be immediately ported back to \MultiChor:
since \MultiChor's representation of \inlinecode{Located} is isomorphic to \inlinecode{Maybe},
the class implementations are straightforward;
all that was missing was confidence that those interfaces were safe to expose, and \minichor gives us that.
For any construct in Haskell to implement these classes is a major usability advantage~\cite[Chapter~4]{haskell-cookbook}.
As an example, a \MultiChor declaration of a variable \inlinecode{filtered} like
\begin{minted}[xleftmargin=30pt,fontsize=\small]{haskell}
do let myFilter :: (Key->Widgit->Bool) -> Key -> [Widgit] -> [Widgit] = ...
       comparator :: Located servers (Key->Widgit->Bool) = ...
       key :: Located workers Key = ...
       values :: Located workers [Widgit] = ...
   filtered <- congruently (transitive workers servers) \un ->
         myFilter (un workers comparator) (un refl key) (un refl values)
   ...
\end{minted}
can be rewritten using "map" (\inlinecode{<$>} from \inlinecode{Functor}) and "splat" (\inlinecode{<*>} from \inlinecode{Applicative}) as
\begin{minted}[xleftmargin=30pt,fontsize=\small]{haskell}
do let ...
   let filtered = myFilter <$> (othersForget workers comparator)
	                   <*> key
			   <*> values
   ...
\end{minted}
In \minichor's case studies this pattern mostly replaces the analogs of \inlinecode{congruently};
it's less clear if this pattern improves or degrades the ergonomics of \inlinecode{locally} and its variants.

We don't present \minichor as "\MultiChor-V2"
because it's advantages or disadvantages, compared to \MultiChor in terms of performance or ergonomics, are ambiguous.
Although precisely assessing the usability of a system like \MultiChor will require a future human subjects study,
it's plausible that that the advantages of the above pattern could more-than-offset the deficits of a \inlinecode{naked}-based system.
For this reason, we suggest including \minichor or a system like it in any such future usability study.


\bibliographystyle{chicago}
\bibliography{refs}
