\chapter{Conclusions and Future Work}
\label{sec:conclusion}

In \Cref{sec:formalism} we introduced a new paradigm for choreographic programming,
showed that it has the basic properties foundational to the field,
and described how to replicate the expressivity and efficiency of other paradigms in this new \emph{conclaves} paradigm.
In \Cref{sec:multichor} we showed how to implement library-level CP with conclaves (and MLVs) in Haskell,
and showed how Haskell's type system enabled CP design patterns like census polymorphism that were previously impossible.
In \Cref{sec:future} we demonstrated that some aspects of the \MultiChor system are extranious;
the CP system \minichor can represent all of the same case-study programs
without using freer monad handlers or having a built-in notion of located values.

While we believe the advantages of \MultiChor to be unique at this time,
we do not expect the field to remain stagnant
and there are critical things \MultiChor choreographies can not do such as recover from communication failure.
Furthermore, our experience observing other people attempt to use \MultiChor suggests
that its flexibility may not suffice, in the minds of prospective users, to justify its corresponding cognitive overhead.
Substantial work remains to do to make choreographic programming an attractive paradigm for industry users.
Our hope at this time is that \MultiChor and \minichor are well built and well positioned as stepping-off points for such future work.

In a few places in this chapter we have suggested immediate changes that could be made to \MultiChor.
For version-2.0 \textit{per se}, we advocate incorporating as many of them as practical
(improved documentation, streamlined user-facing API, instances of \inlinecode{Functor} \textit{etc} for \inlinecode{Located})
while keeping fundamental systems like the proof-witnesses intact.
Further future work constitutes a substantial and open-ended research campaign that might be tackled in any order:
\begin{itemize}
	\item Compose a formal model of \minichor and a comparable select-\&-merge model,
		with the goal of showing an equivalence between them.
	\item Conduct a structured study comparing the usability of \MultiChor, \minichor,
		and other relevant systems that target real industry use.
	\item Conduct a structured study comparing the performance of \MultiChor, \minichor,
		and other relevant systems that target real industry use.
	\item Develop a system for lazy choreographies as discussed in \Cref{sec:minichor-implications}.
	\item Augment \MultiChor with tools for modeling and recovering from communication failures.
\end{itemize}

In the meantime, we hope that both researchers and interested industry practitioners
will see the work presented here as the cutting edge of applied choreographic programming,
and as a suitable foundation for further development.




